\section{Introduction}
Long-lifetime, reliable electricity sources will be vital a vital engineering
component of future long-term space missions. Nuclear energy systems are
promising in their ability to deliver large amounts of electricity for long
mission timelines. Traditionally, radioisotope thermoelectric generators (RTGs)
have been used to provide electrical power for deep space missions. Planned
future space missions such as a manned mission to Mars will require much more
power than RTGs currently provide. System mass is a strong driver of launch
costs and feasability. This work optimizes the design of a nuclear
fission reactor to power a Supercritical CO$_2$ Brayton Cycle as part of an
overall mass minimization for the entire electrical power system. The reactor
mass contribution to the overall system mass is modeled with a surrogate reactor
model to bypass traditional (and computationally slow) reactor physics tools.
This surrogate reactor mass model is the focus of the project.

\section{Scope}
The scope of this project differs from traditional reactor design concepts. The
purpose of this work was not to create a mature concept reactor design. Rather,
the surrogate reactor mass model was the focus of the project effort. This mass
model was used to inform a valid reactor design in conjunction with the power
cycle design. Therefore, many specific details about the reactor design have not
been included in this analysis. The chosen reactor design has been modeled
sufficiently to prove validity, with the goal that a team of engineers could
bring the design to production within a reasonable amount of time.
