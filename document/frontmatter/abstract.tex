Long-term space exploration to Mars and other parts of the solar system will
require significant amounts of electrical power. Nuclear reactors have long been
considered to meet these power needs. Minimizing the total system mass of
electrical systems is crucial to minimize launch costs. A surrogate reactor
model was developed to optimize the mass of a nuclear reactor for space
applications. The model was designed to be used in conjunction with power cycle
component mass models to explore the tradeoff between component masses of 
with the goal of minimizing the total mass of the system. The
reactor mass model needed to rapidly execute in order to be useful in an
optimization algorithm. To this end, a surrogate mass model was developed that
did not rely on high-fidelity reactor physics and finite element analysis tools.
The model included a reactivity constraint to support a 10 year mission life and
a thermal constraint to ensure fuel integrity.

The reactor mass model was coupled to an external power cycle model through flow
inputs. Coolant flow conditions at the inlet and outlet of the core define
thermophysical properties and core thermal power requirements that were used by
the thermal hydraulic model to estimate required volumetric power densities in
the reactor core.

The reactivity constraint was modeled as a beginning of life (BOL) excess reactivity
target. The target was chosen using a large dataset of depletion calculations
and was deemed sufficient to ensurve the optimized designs had sufficient excess reactivity
to sustain 10 years of full-power operation. Important neutronics and operational
parameters were also tested in this dataset to determine which parameters were
most effective for use in a reduced-order reactivity model.

A reduced-order surrogate reactivity model was developed to constrain a
minimum-mass reactor to a target BOL reactivity. The reduced-order model was
created from a large dataset of criticality calculations using MCNP6.1. This
model was used to constrain the reactor geometry in the thermal hydraulic model.
Thousands of MCNP \keff calculations were performed to generate a relationship
between core radius, fuel fraction, reflector thickness, and \keff. This dataset
was represented with trilinear interpolation in order to develop a
mass-minimized relationship between core radius and fuel fraction that met the
target excess reactivity.

The reactor mass model was ultimately a neutronically-constrained thermal
hydraulic model. The reactivity model was used to constrain the core radius as
a function of fuel fraction in the thermal hydraulic model. The thermal
hydraulic model used a root-finding routine to find a fuel fraction that met
the required thermal input from an external power cycle model. The model used 1D
heat transfer models to determine the fuel fraction that, combined with a constrained
core radius, could generate the required thermal input. The result was a
mass-optimized, fully constrained reactor design that met coolability and reactivity
requirements for a 10 year mission and the given power cycle inputs.
