Long-range space exploration to Mars and other parts of the solar system will
require significant amounts of electrical power. Nuclear reactors have long been
considered to deliver these power needs. Minimizing the total system mass of
electrical systems is crucial to minimize launch costs. A surrogate reactor
model was developed to optimize the mass of a nuclear reactor for space
applications. The model was designed to be used in conjunction with power cycle
component mass models to explore the tradeoff between reactor and power cycle
component mass with the goal of minimizing the total mass of the system. The
reactor mass model needed to rapidly execute in order to be useful in an
optimization algorithm. To this end, a surrogate mass model was developed that
did not rely on high-fidelity reactor physics and finite element analysis tools.
The model included a reactivity constraint to ensure th

The reactor mass model was coupled to an external power cycle model through flow
inputs. Coolant flow conditions at the inlet and outlet of the core define
thermophysical properties and core thermal power requirements that were used by
the thermal hydraulic model to estimate achievable volumetric power densities in
the reactor core.


