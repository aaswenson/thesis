\section{Critical Core Radius}

Neutronic viability is an important part of reactor design. The figure of merit
for neutronics is EOL \keff. The reactor must be able to sustain a fission
chain reaction from startup to shutdown on the final day of the 10 year mission.

The criticality requirement is the second of two major criteria for a valid reactor design and an
important constraint in the thermal hydraulic modeling process. As a result,
modeling criticality of various reactor configurations demands its own chapter
in this thesis. Criticality modeling started with depletion modeling, reported
in section \ref{neutronics_sweeps}. The purpose and conclusion of the depletion
modeling was that thermal power and small geometric features were of low
importance when predicting EOL \keff. In addition, higher enrichment will always
be prefferred to lower enriched uranium.

The thermal-hydraulic modeling scheme
reported in section \ref{reactor_mass_model} requires a constraint for the core
radius. A relationship between core radius and fuel fraction for a target
value of \keff was developed using traditional reactor physics tools. In lieu of
full depletion modeling, BOL \keff was modeled and the \keff target adjusted to
ensure sufficient excess reactivity for 10 years of full-power operation.

There were three major components to developing the critical core radius
constraints, determining a value for BOL excess reactivity that would satisfy
EOL \keff > 1, optimizing the reflector radius for each fuel type to minimize
overall mass, and generating the relationship between fuel fraction and the
required core radius to meet the target \keff value. Critical core radius is a
bit of a misnomer, since the target of this analysis was a BOL excess
reactivity.

\subsection{Optimized Reflector Thicknesses}
Reflectors are used in nuclear reactor design to reduce neutron leakage from the
reactor core. Reflectors can reduce required fuel mass by reflecting neutrons
back into the core and preventing them from being lost to leakge. However,
adding reflector thickness to the outside of the core adds mass to the overall
reactor system. There exists a tradeoff between mass of fuel and mass of
reflector and an optimal combination of the two. The purpose of this analysis
was to optimize reflector thickness as a multiplier of the overall core radius.
This analysis was done for each fuel-coolant combination.

\subsubsection{Modeling Methods}
