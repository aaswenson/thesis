\chapter{Critical Core Constraint}\label{ch:crit_radius}
Neutronic viability is naturally an important part of reactor design.The figure of merit
for neutronics is EOL \keff. The reactor must be able to sustain a fission
chain reaction from startup to shutdown on the final day of the 10 year
mission. Nuclear reactor design processes traditionally start with an in-depth
neutronic design stage, producing a fully-fledged neutronics design before
moving on to in depth consideration of other fields such as materials corrosion
and thermal hydraulics. This chapter will discuss the methods
used to develop a surrogate reactivity model.

The critical core constraint was developed to ensure coolable reactors were also
neutronically valid. Chapter \ref{ch:mass_model} describes the thermal hydraulic
model of the core. The critical core constraint was used to fix the core radius
as a function of fuel fraction in the thermal model. This ensured the thermal
model was producing reactors with enough excess reactivity to operate for 10 years.

\section{Modeling Techniques}
There were three major steps to developing the critical core radius
constraints. The first component was determining a value for BOL excess reactivity that would satisfy
EOL \keff > 1. The second step was generating the relationship between fuel fraction and the
required core radius to meet the target \keff value. Finally, for each fuel fraction, an
optimized reflector thickness was also calculated to minimize total core mass.
It is worth noting that throughout this analysis, critical core radius is a
bit of a misnomer since the target of this analysis was a BOL excess
reactivity.

This process will be described in full for a single reactor core (\uox-\codiox). The process
was performed for additional combinations of coolants and fuels and in theory,
could be applied to future reactor concepts. Initially, the \uox-\codiox and the
UW-\codiox reactors were used to develop the model. One of the goals of the
model was to be design-agnostic. In the future, another fuel form or coolant could be used
with this workflow to replicate a mass optimization on a new design.

\subsection{Homogeneous Reactivity Models}
The \keff response to three modeled parameters was generated on a grid using
MCNP6.1 models. The modeled parameters were core radius, fuel fraction, and
reflector thickness as a multiplier of core radius. Core radius and fuel
fraction were chosen in the feature selection analysis discussed in Chapter
\ref{ch:sweeps}. Reflector thickness was added in order to minimize the 
critical reactor mass. 
The models were homogeneous geometries consisting of
a core region (fuel, cladding, and coolant), a reflector region (graphite), a
pressure vessel (SS-304), and finally the surronding Martian regolith. The top of the
reflector was buried 1 m below the surface of the Martian regolith. Figure
\ref{fig:homog_model} shows an example reactor geometry.

\begin{figure}[h]
    \centering
    \includegraphics[width=4in]{../images/crit_model_geom.png}
\caption{Example reactor MCNP6.1 model.}
\label{fig:homog_model}
\end{figure}

The reactor model and surronding Martian soil was modeled in a sphere. Vaccuum
boundary conditions were imposed on the edge of the sphere. The radius of the
sphere was set sufficiently large so that escaping neutrons would be unlikely to
return to the reactor.

Every \keff calculation was performed at room temperature. Neutron absorption
due to doppler broadening in 
\urantwo was assumed negligible due to high enrichment. A higher energy
spectrum due to no moderation also increase the resonance escape probability.
This assumption was checked in later neutronics verification calculations. 
In the end, BOL excess reactivity was chosen such that it overcame any 
temperature-induced reactivity losses. 

\subsection{Excess Reactivity Margin}
As discussed in Section \ref{ch:sweeps}, EOL \keff is weakly dependent on
thermal power. This meant that EOL \keff could be modeled as BOL \keff with enough
excess reactivity to account for depletion. With a target BOL reactivity,
depletion neutronics could be replaced by simple \keff calculations. This
was important because traditional depletion tools are too computationally
expensive to explore the desired reactor parameter space. Excess reactivity was
chosen based on the reactivity change in each of the 3901 depletion calculations. Figure
\ref{fig:delta_k_eol} shows the reactivity swing $(\Delta k$) for every core
with 100 kg < core mass < 500 kg. These mass bounds were chosen by initial thermal analysis
showing the reactor design would fall in this mass domain.

\begin{figure}[h]
    \centering
    \includegraphics[width=4in]{../images/dK_vs_mass.png}
\caption{Reactivity swing over reactor lifetime}
\label{fig:delta_k_eol}
\end{figure}

A BOL \keff of 1.01 was deemed sufficient to sustain the fission chain reaction
through 10 years of full-power operation. This \keff target was used to define a
critical core radius, and to optimize the reflector thickness multipliers
(discussed in the following sections).

\section{Modeling \keff in 3D Parameter Space}\label{sec:crit_model}
In order to constrain the thermal hydaulics equations for the 
thermal hydraulic reactor model, a critical radius relationship to fuel fraction was developed. 
This section discusses the methods used to derive a relationship between \keff,
core radius, fuel fraction and, reflector multiplier. The workflow to
optimize the mass of the \keff relationship subject to a reactivity constraint
will be discussed as well. The following method is discussed for the \uox-\codiox
reactor configuration. The workflow was eventually applied to all four reactor
designs (see Appendix \ref{ch:appendix-a}.

\subsection{BOL Reactivity Modeling}
Three predictor variables were explored to model BOL \keff: core radius,
fuel fraction, and reflector mulitplier. The reflector multiplier is used to
define the radius of the reflector as a function of core radius. Equation
\ref{eq:ref_mult} shows the reflector radius as a function of core radius and
reflector multiplier.

\begin{equation}
    \label{eq:ref_mult}
    R_{refl} = mR_{core}
\end{equation}

Core radius, fuel fraction, and reflector multiplier all impact \keff.
Increasing the core radius adds more fuel mass and decreases neutron leakage 
from the core. Increasing fuel fraction also increases fuel mass, increasing
absorption cross sections in the fuel and decreasing leakage from the core.
Finally, adding reflector thickness prevents neutrons from leaking from the
core. A 3-dimensional grid of sampling data was generated to model each
predictor's impact on \keff. For each
point in 3d (radius, frac, multiplier) space, a \keff calculation was
performed using MCNP6.1.


\subsection{Parameter Ranges}
Initial scoping calculations were performed to determine appropriate ranges for
each of the three parameters (radius, frac, and multiplier). After the scoping,
9900 kcode
calculations were run on a grid, evenly sampling in each dimension.
Table \ref{tab:bol_criticality_1000} shows the sampled domain for the initial
scoping calculation.

\begin{table}[h]
  \centering
  \caption{Criticality Sampling Domain}
  \begin{tabular}{lll}
    \toprule
     Parameter               & Range          & Samples\\ 
    \midrule                                  
     Core Radius             & 10 - 50 [cm]   & 30 \\
     Fuel Fraction 		     & 0.1 - 0.95 [-] & 10 \\
     Reflector Multiplier    & 0.001- 1.5 [-] & 33 \\
  \end{tabular}
  \label{tab:bol_criticality_1000}
\end{table}

Each of the 9900 data points were used to generate a \keff response over a 3D
domain of core radius, fuel fraction, and reflector multiplier. This data set was used to
generate a set of mass-optimized reactors constrained to a target BOL \keff of
1.01. Section \ref{sec:crit_rad_search} discusses the algorithm used to generate
the minimized, constrained reactor masses.

\section{Constrained Critical Mass Optimization}\label{sec:crit_rad_search}
A relationship between \keff and the three input
parameters (r, f, m) was developed, such that:

\begin{equation}
    k_{eff} = k(r, f, m)
\label{eq:gen_keff}
\end{equation}

Where $r$ was core radius, $f$ was fuel fraction, and $m$ was reflector multiplier. This
function was used to constrain the models to the required \keff target and
minimize the total critical reactor mass.

\subsection{Reactivity Interpolation in 3D Space}
Linear interpolation was used to develop a function for \keff to be used to
constrain the core geometry. SciPy's 'RegularGridInterpolator' package was used
to generate a callable Python function. This function used the grid data
generated in Section \ref{sec:crit_rad_search} to return \keff as a function of
r, f, and m. Trilinear interpolation was used to estimate the \keff response.
Trilinear interpolation performs a linear combination of the 8 nearest points in
a cube surronding the desired point of interpolation. Linear interpolation is
only continuous to the zeroth derivative, but for large data sets, provides a
fast-executing function to model \keff.

Scipy's grid interpolator can be stored as a Python object and rapidly executed like any
Python function, making it useful as the target function for the mass
optimization. Despite numerous pratical advantages for using a trilinear
interpolator, there is systemic error introduced into the optimization by
representing nonlinear curves as piecewise-linear functions. The grid
interpolator was tested in order to understand the magnitude of this induced
error. A testing set of data was generated in the same domain space as the data
used to train the interpolator. This data was generated using Latin Hypercube
Sampling instead of being spaced evenly on the grid. LHS ensured that every
point would check the interpolator with data that did not come directly from
the grid. The testing set was 500 \keff samplings. For each \keff result in the
testing data, an interpolated result was generated using the interpolation
function that was created from the training data. The absolute error was calculated
between each test \keff and its corresponding interpolated \keff. Figure
\ref{fig:interp_check} shows the distribution of the absolute error. Trilinear
interpolation was deemed adequate to model reactivity.

\begin{figure}[h]
    \centering
    \includegraphics[width=4in]{../images/check_interp.png}
\caption{Reactivity discrepancy between interpolated and sampled data.}
\label{fig:interp_check}
\end{figure}

\subsection{Critical Core Radius}
Core radius was used as the free variable to meet the \keff constraint target. For
a given fuel fraction and reflector multiplier, a core radius exists that meets
the target \keff. At the heart of this workflow is Equation \ref{eq:gen_keff},
and its representation as a trilinear interpolation function. This function was
used to interrogate the \keff response to varying core radii at fixed fuel
fractions and reflector multipliers. SciPy's 'minimize\_scalar' funciton was used to
perform this search. The goal of the search was to minimize the absolute error
between the interpolated \keff and the target \keff (k$_0$ = 1.01).

\begin{equation}
    min( (k(r, f, m) - k_0)^2 )
    \label{eq:crit_rad}
\end{equation}

For a fixed fuel fraction and reflector multiplier, 'minimize\_scalar' performs a
bounded optimization to return a valid core radius to meet the target \keff
constraint. This routine was executed in a Python function called 'crit\_radius'.
This function was used in multiple steps to optimize the reactor mass.

\subsubsection{Critical Mass Optimization}
With a routine in place to constrain core radius to a target \keff, an
optimization to minimize mass was performed. The goal of the critical mass model
was to provide the thermal hydraulic model with a critical core radius
constriaint. For each
fuel fraction, the thermal hydraulic mass model needed a core radius that
represented a critical core mass. In addition to a critical mass, the critical
core mass should be optimized to minimize total reactor mass. 
The reflector multiplier (m) was used as the free variable to optimize overall core mass. 

Reflectors impact neutronics by reflecting neutrons back into the core. This
reflection reduces leakage and improves the neutron economy, as a result,
reflected cores require less fuel mass than bare cores. Increasing the
thickness of the reflector helps to a point, but eventually adding more reflector
thickness does not improve neutronic performance. Neutron leakage of course, is only
one part of the neutron life cycle, there are diminishing neutronic returns to
reflector thickness. In addition to improving the neutron economy, increasing
the reflector size increases the mass of the overall reactor. The tradeoff
between reflector mass and fuel mass was explored with the objective of
minimizing total reactor mass. 

Reflector thickness optimization was performed using the previously-discussed critical radius
workflow. The goal of the optimization was to produce a
critical core, with an optimized reflector thickness to minimize the
overall reactor mass. For every fuel fraction, a set of 20 multipliers was
checked from 0 to 0.12. Equation \ref{eq:crit_rad} returns a critical radius for
each combination of fuel fraction and reflector multiplier. This critical radius
was then used to define the critical core mass. The critical core mass includes
the mass of the fuel, coolant, cladding, reflector, and pressure vessel.

\begin{equation}
    M_{total} = f(r, f, m)
\end{equation}

The total mass (fuel, reflector, pressure vessel) for each critical
reactor was fitted against the reflector multipliers with a 2nd degree
polynomial. This polynomial was used as a target function for SciPy's
'minimize\_scalar' to minimize the core mass in reflector multiplier space.

\begin{equation}
    min( M_{total}(r, f, m) )
\end{equation}

The goal of the reflector optimization was to minimize the overall reactor mass
as a function of reflector multiplier for a fixed fuel fraction and a core
radius constrained to a \keff target. The optimization was performed over a range
of fuel fractions, for each fuel fraction, the optimized reflector was used to
generate a critical core radius that optimizes total reactor mass.


\subsubsection{Critical Mass Optimization Results}

The critical mass optimization in reflector multiplier space was performed for
each reactor configuration. The following is the results of the \uox fueled,
\codiox cooled reactor.

\begin{figure}[h]
    \centering
    \includegraphics[width=4in]{../images/mass_mult_064.png}
\caption{Total reactor mass optimization for fuel fraction of 0.64}
\label{fig:mass_mult_one}
\end{figure}

Figure \ref{fig:mass_mult_one} shows the result of this optimization for one
fuel fraction, 0.64 for the UO2-CO2 reactor configuration. An important takeaway here, is the lack of curvature in
this plot. The reactor mass for this configuration was relatively insensitive to
reflector thickness. When compared to the total power cycle mass (on the order
of 1000 kg), the 2 kg difference in reactor mass was deemed insignificant.
This relative insensitivity to reflector mass was due to the HEU fuel providing a much
stronger absorption cross section, resulting in less neutron leakage from the core. In
addition to strong fuel absorption, the reactor was buried in the Martian
regolith, some reflection can be expected from the soil around the reactor.
The standard deviation of the critical masses across reflector multiplier space
further suggests that reflector thickness was unimportant for the mass
optimization. The standard deviation of the mass ($m$) was calculated across $N$
multipliers for each fuel fraction.

\begin{equation}
    s = \sqrt{\frac{\sum_{i=0}^{N}(m_i - \bar{m})^2 }{N}}
\end{equation}

\begin{figure}[h]
    \centering
    \includegraphics[width=4in]{../images/mass_std_uo2_co2.png}
\caption{Standard deviation of reactor mass in reflector multiplier space}
\label{fig:mass_std_co2_uo2}
\end{figure}

The standard deviation of the mass across the range of tested reflector
multipliers was very small relative to the mass of the overall power cycle,
which was on the order of 1000 kg. As a result, the reflector multiplier was fixed to an
average value of 0.102 for the \uox-\codiox configuration. 
This average value was used to generate a critical radius
curve as a function of fuel fraction. This curve was used by the thermal
hydraulic reactor mass model to constrain the radius of the core for a given
fuel fraction. 


\subsubsection{Choosing a fitting function for critical core radius}
The reactor mass model needed a functional relationship between fuel fraction
and critical core radius. This was achieved with polynomial curve fitting.
Equation \ref{eq:polyfit} describes the target polynomial for the critical
radius fits, where n is the degree of fit.

\begin{equation}
    R_{critical} = \sum_{i=0}^n A_if^i
    \label{eq:polyfit}
\end{equation}

The choice of order of the polynomial was very important to accurately represent the
critical core radius. Reactivity was very sensitive to small mass changes in mass, as
a result, an underfitted polynomial could return reactor masses that do not
result in the target BOL \keff. A simple study was conducted to make an informed
choice for the degree of fit. The study swept orders of polynomials from 2 to 9.
For each order, the reactor thermal hydraulic was used to generate an optimized reactor
design. The parameters of each core were modeled with MCNP6.1 to verify the
reactivity of the design. This was done for both the \uox-\codiox and
the UW-\codiox reactor configurations. Figure \ref{fig:fit_order_test} shows the
result of this study.

\begin{figure}[h]
    \centering
    \includegraphics[width=4in]{../images/poly_order_test.png}
\caption{Testing critical radius fit polynomials.}
\label{fig:fit_order_test}
\end{figure}

Polynomials less than 4th order were clearly underfitted and misrepresented the
critical radius relationship. A 7th degree polynomial was chosen to represent the
data. Figure \ref{fig:core_r_co2_uo2} shows the fitted curve and the
data generated by the critical radius algorithm and the underlying 3D
interpolation function.

\begin{figure}[h]
    \centering
    \includegraphics[width=4in]{../images/uo2_co2_core_r.png}
\caption{Critical radius dependence on fuel fraction.}
\label{fig:core_r_co2_uo2}
\end{figure}

The 7th order fit coefficients are shown below in Table
\ref{tab:uo2_co2_fit_coeffs}.

\begin{table}[h]
  \centering
  \caption{\uox-\codiox 7th order fit coefficients}
  \begin{tabular}{cc}
    \toprule
     Degree & Coef\\ 
    \midrule                                  
    A$_0$  &  70.924\\
    A$_1$  &  -457.719\\
    A$_2$  &  2065.779\\
    A$_3$  &  -5960.595\\
    A$_4$  &  10782.565\\
    A$_5$  &  -11759.310\\
    A$_6$  &  7028.420\\
    A$_7$  &  -1761.031\\
  \end{tabular}
  \label{tab:uo2_co2_fit_coeffs}
\end{table}

\section{Critical Core Constraint Summary}
A critical radius constraint was developed to support the thermal hydraulic
reactor model. Core radius was fixed to reach a target \keff as a function of
fuel fraction. Core radius and fuel fraction were chosen as important neutronic
variables based on analysis discussed in Chapter \ref{ch:sweeps}. A reflector
thickness multiplier was used to minimize the total mass of a critical reactor.
The optimized multipliers were found to be constant as a function of fuel
fraction and were modeled as such. Data was generated to represent critical core
radii as a function of fuel fraction. These data were fit to a 7th
degree polynomial that was used by the thermal hydraulic model to constrain the
core radius. The implementation of this constraint will be discussed in the next
chapter.
