\chapter{Power System Overview}
As discussed in the introduction, the modeled electrical power system is a 40
kWe, recuperated Brayton cycle driven by a direct-cooled nuclear fission reactor. The goal
of the project was to develop a mass model for the major components of the
electrical power system. These mass models were then used to explore the
tradeoff in performance and mass between each of the systems. In general terms,
an efficient power conversion cycle requires a smaller thermal input, but is
more massive. Conversely, a less massive power cycle has lower efficiency,
requiring a larger thermal input. This section will provide context to the
development of the reactor mass model.

\section{Cycle Description and Mass Modeling}
The modeled cycle was a direct Brayton cycle coupled to a fission reactor. The
cycle was recuperated for increased efficiency. A cycle diagram is shown below
with the three modeled components. Figure \ref{fig:power_cycle} shows the power
cycle diagram for this system.

\begin{figure}[h]
    \centering
    \includegraphics[width=5in]{../images/power_cycle.png}
\caption{Power cycle diagram}
\label{fig:power_cycle}
\end{figure}

Three main power system components were modeled for
mass; the radiator, recuperator, and the reactor. This thesis focuses on the
reactor mass model. The goal of this work was to deliver a mass
model to a colleague in order to optimize the overall power cycle mass
performance. The colleage developed mass models for the radiator and recuperator
as well as a cycle model to model the thermodynamic performance of the system.
Turbine and compressor masses were not modeled, these were off-the-shelf
components. Even if the compressor and turbine were custom-designed, their
negligible mass contribution to the power cycle mass would make optimizing their
design relatively fruitless.

\subsection{Reactor Coupling}
The interface between the reactor and the rest of the power cycle were flow
conditions. The reactor thermal output was determined by the required enthalpy
input to match the power cycle performance and meet the 40 kWe project
requirement. The flow conditions were important for two reasons, they dictated
the thermal power requirements for the core, and they impacted the convective
heat transfer from the reactor cladding to the coolant. The flow properties
provided by the power cycle model were:

\begin{enumerate}
    \item inlet/outlet temperature
    \item inlet/outlet pressure
    \item mass flow rate
    \item thermal power
\end{enumerate}

These conditions were taken as inputs to the power cycle model. For the sake of
cycle optimization, the model returned a reactor mass that matched the flow
conditions and thermal power requirements. For future neutronics modeling, the
model could also return geometric core parameters (radius, fuel fraction,
reflector thickness, etc.). 

\section{Model Development}
The reactor and power cycle models were developed independently and combined
near the end of the project to model the entire system. While the reactor model
was under development, a simplified analytic model was used in place of the
reactor model to return an estimate of reactor mass. While the power cycle model
was under development (in tandem with the reactor model), a set of assumed flow
inputs were used to simulate a power cycle thermal load. The reactor model was
developed in Python and the power cycle model was developed in MATLAB. This
unintentional mismatch in development choices proved interesting when it came
time to merge the models! Despite this challenge, the reactor model was smoothly
imported into the MATLAB power cycle model and the two models worked seamlessly
to model the overall cycle mass.
