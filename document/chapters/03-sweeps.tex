\chapter{Neutronics Feature Selection} \label{ch:sweeps}
A valid reactor design must meet certain neutronics constraints. The main
neutronics constraint is reactivity. The reactor must be able to sustain a chain
reaction from startup, to the final second of the 10 year lifetime. End of Life
(EOL) reactivity was the target metric for the neutronic design of the core.
Traditional reactor design processes will spend a long time tweaking neutronics
parameters to meet certain system requirements. The total mass optimization
needed a rapidly excecutable model that could respond to various inputs.
This meant it was important to understand how a reactor's neutronic
performance is impacted by various input and design parameters in order to
construct a reduced-order neutronics model that could be used in the mass
optimization routine. A 5-dimensional 
sampling was conducted in order to select important predictors for EOL
reactivity.

\section{Parametric Study of Neutronics Parameters}
Homogeneous MCNP6.1 depletion models were used to analyze the EOL \keff response
to neutronics parameters (predictors). The core region was homogenized and
surrounded by a graphite reflector. A large set of 5-dimensional predictors and
EOL \keff results was produced to investigate the neutronics parameter space.
These predictors and their results helped develop an understanding of the
reactor response to important design and operational parameters. High Throughput
Computing capabilities at UW were used to perform 3901 depletion calculations
with MCNP6.1. Each model represented a unique sampling in every dimension using
the Latin Hypercube Sampling (LHS) technique\citep{LHS}. The LHS technique ensured even
sampling for every dimension. The sampled and fixed dimensions/parameters are
shown in Table \ref{tab:lhs_sweep_vars}.

\begin{table}[h]
  \centering
  \caption{Homogeneous Geometry and Depletion Parameters}
  \begin{tabular}{ll}
    \toprule
     Core Radius                		   & 10 - 50 [cm] \\
     Fuel Enrichment 					   & 20\% - 90\% $^{235}$U\\
     Coolant Channel Radius                & 0.5 - 1 [cm] \\
     Fuel Pitch to Coolant Channel D.      & 1.1-1.6 [-]\\
     Thermal Power						   & 80-200 [kW]\\
     Reflector Thickness				   & 15 [cm]\\
     Core Aspect Ratio					   & 1 [-] \\
     Fuel Temperature					   & 300 [K]\\
     Reactor Physics Code, Data			   & MCNP6.1, ENDF-7.2
  \end{tabular}
  \label{tab:lhs_sweep_vars}
\end{table}

\section{Parametric Study Results}
The target metric for the neutronics sampling was EOL \keff $\geq$ 1.0. In
order to determine the dependence of EOL \keff on each swept parameter in Table
\ref{tab:lhs_sweep_vars}, EOL \keff was plotted against the parameters.

\subsection{Uranium 235 Mass}
The strongest predictor of EOL \keff is the mass of \uran
in the system at beginning of life (BOL). The other predictors discussed 
in this chapter affect EOL \keff
through their impact on \uran mass. Figure \ref{fig:eol_keff_vs_235_mass} 
shows the dependence of EOL \keff on \uran mass.

\begin{figure}[h]
    \centering
    \includegraphics[width=3in]{../images/keff_vs_mass_235.png}
\caption{EOL \keff dependence on \uran mass}
\label{fig:eol_keff_vs_235_mass}
\end{figure}

This conclusion was useful for modeling a reactor's mass. Since EOL \keff is
dependent on \uran mass, for a fixed enrichment, EOL \keff can be modeled by
using the few parameters that affect reactor mass.

\subsection{Core Radius}
Core radius has a strong impact on EOL \keff. Fuel mass is directly correlated
to fuel fraction and core volume. For a fixed aspect ratio of 1, core volume is
a cubic function of core radius. This makes EOL \keff strongly dependent on core
radius. The volume to surface area ratio is a positive function of core radius
hence, neutron leakge is reduced in larger cores. Figure
\ref{fig:eol_keff_vs_r_core} shows the correlation between core radius and EOL
\keff.

\begin{figure}[h]
    \centering
    \includegraphics[width=3in]{../images/keff_vs_mass_235_core_r.png}
\caption{EOL \keff dependence on core radius.}
\label{fig:eol_keff_vs_r_core}
\end{figure}

Figure \ref{fig:eol_keff_vs_r_core} displays the strong relationship between
core radius and fuel mass. Increasing fuel mass increases EOL \keff.

\subsection{Fuel Enrichment}
Fuel enrichment directly affects \uran mass in nuclear fuel. Higher enrichments
yield more \uran mass for a fixed mass of fuel. Figure
\ref{eol_keff_vs_mass_enrich} shows how increasing enrichment increases \uran
mass and EOL \keff. For fixed \uran masses, increasing fuel mass also impacts
reactivity. For fixed \uran masses, higher enriched cores have greater \uran
density. This increased density increases EOL \keff.

\begin{figure}[h]
    \centering
    \includegraphics[width=3in]{../images/keff_vs_mass_235_enrich.png}
\caption{EOL \keff dependence on fuel enrichment}
\label{fig:eol_keff_vs_mass_enrich}
\end{figure}

\subsection{Fuel Pitch to Coolant Channel Diameter}
The fuel pitch to coolant channel diameter ratio (PD) also directly impacts
\uran mass in fuel. PD affects volumetric fuel fraction. Figure
\ref{fig:eol_keff_vs_PD_mass} shows the impact of PD on \uran mass and EOL
\keff. In addition to increasing \uran mass, PD impacts uranium density.
Reactors with larger PD ratios have greater \uran density which reduces neutron
leakage from the core and increases reactivity.

\begin{figure}[h]
    \centering
    \includegraphics[width=3in]{../images/keff_vs_mass_235_PD.png}
\caption{EOL \keff dependence on fuel pitch to coolant channel ratio}
\label{fig:eol_keff_vs_PD_mass}
\end{figure}


\subsection{Coolant Channel Radius}
EOL \keff is independent of coolant channel radius. Channel radius had no impact
on fuel mass and small geometric features
had little importance for high energy neutrons. Figure
\ref{fig:eol_keff_vs_r_cool} shows the impact of coolant channel radius.

\begin{figure}[h]
    \centering
    \includegraphics[width=3in]{../images/keff_vs_cool_r.png}
\caption{EOL \keff dependence on coolant channel radius}
\label{fig:eol_keff_vs_r_cool}
\end{figure}


\subsection{Thermal Power}
Initially, the result of greatest interest was the mass dependence on thermal
power, because thermal power was the reactor-power cycle coupling parameter.
Figure \ref{fig:eol_keff_vs_power} shows the relationship between thermal power
and EOL \keff.

\begin{figure}[h]
    \centering
    \includegraphics[width=3in]{../images/keff_vs_power.png}
\caption{EOL \keff dependence on thermal power}
\label{fig:eol_keff_vs_power}
\end{figure}

As shown in Figure \ref{fig:eol_keff_vs_power}, EOL \keff is independent of core
thermal power. This result is not surprising considering the depletion rates in
the core. Assuming 1 MWd/gU is an achievable burnup, the reactor depletes
approximately 1000 g of uranium over 10 years of full power operation (at 200
kW). The depletion mass
of uranium is negligigle compared to the mass of uranium required for the
reactor to be critical at BOL. Thermal power is not a strong predictor of EOL
criticality.


\section{Parametric Study Conclusion}
The purpose of this parametric study was to perform a rudimentry feature
selection. Feature selection is a technique used in machine learning and
statistics to select relevant variables (features) in a data set to simplify the
modeling of the data set. The parametric study helped identify key predictors
for the neutronic performance of a reactor design.

The figure of merit for neutronics performance was EOL \keff. The reactor must
sustain a fission chain reaction throughout the mission lifetime. In order to
ensure the reactor mass model generates a neutronically viable core, it was
necessary to determine on which input parameters EOL \keff is most strongly
dependent. To support this analysis, 3901 MCNP6.1 depletion calculations were
performed over a 5-dimensional sample space with Latin Hypercube Sampling
techniques to ensure even sampling. The conclusion of this work is that fissile
material mass is the most important parameter to predict EOL \keff. Another
important conclusion was that EOL \keff is relatively independent of thermal power. This
meant EOL \keff could be estimated as BOL \keff adjusted for marginal burnup.
This conclusion was useful to constrain reactor parameters and will be utilized
in Chapter \ref{ch:crit_radius} to justify a relationship between core fuel fraction
and a required core radius to ensure EOL \keff > 1.
