\chapter{Neutronics Feature Selection} \label{ch:sweeps}
A valid reactor design must meet certain neutronics constraints. The main
neutronics constraint is reactivity. The reactor must be able to sustain a chain
reaction from startup, to the final second of the 10 year lifetime. End of Life
(EOL) reactivity was the target metric for the neutronic design of the core.
Traditional reactor design processes will spend a long time tweaking neutronics
parameters to meet certain system requirements. The total mass optimization
needed a rapidly excecutable model that could respond to various inputs.
This meant it was important to understand how a reactor's neutronic
performance is impacted by various input and design parameters in order to
construct a reduced-order neutronics model that could be used in the mass
optimization routine. A 5-dimensional 
sampling was conducted in order to select important predictors for EOL
reactivity.

\section{Parametric Study of Neutronics Parameters}
Homogeneous MCNP6.1 depletion models were used to analyze the EOL \keff response
to neutronics parameters (predictors). The core region was homogenized and
surrounded by a graphite reflector. A large set of 5-dimensional predictors and
EOL \keff results was produced to investigate the neutronics parameter space.
These predictors and their results helped develop an understanding of the
reactor response to important design and operational parameters. High Throughput
Computing capabilities at UW were used to perform 3901 depletion calculations
with MCNP6.1. Each model represented a unique sampling in every dimension using
the Latin Hypercube Sampling (LHS) technique\citep{LHS}. The LHS technique ensured even
sampling for every dimension. The sampled and fixed dimensions/parameters are
shown in Table \ref{tab:lhs_sweep_vars}.

\begin{table}[h]
  \centering
  \caption{Homogeneous Geometry and Depletion Parameters}
  \begin{tabular}{ll}
    \toprule
     Core Radius                		   & 10 - 50 [cm] \\
     Fuel Enrichment 					   & 20\% - 90\% $^{235}$U\\
     Coolant Channel Radius                & 0.5 - 1 [cm] \\
     Fuel Pitch to Coolant Channel D.      & 1.1-1.6 [-]\\
     Thermal Power						   & 80-200 [kW]\\
     \hdashline
     Reflector Thickness				   & 15 [cm]\\
     Core Aspect Ratio					   & 1 [-] \\
     Fuel Temperature					   & 300 [K]\\
     Reactor Physics Code, Data			   & MCNP6.1, ENDF-7.2
  \end{tabular}
  \label{tab:lhs_sweep_vars}
\end{table}

\section{Parametric Study Results}
The target metric for the neutronics sampling was EOL \keff $\geq$ 1.0. In
order to determine the dependence of EOL \keff on each swept parameter in Table
\ref{tab:lhs_sweep_vars}, EOL \keff was plotted against the parameters.

\subsection{Uranium 235 Mass}
The strongest predictor of EOL \keff is the mass of \uran
in the system at beginning of life (BOL). The other predictors discussed 
in this chapter affect EOL \keff
through their impact on \uran mass. Figure \ref{fig:eol_keff_vs_235_mass} 
shows the dependence of EOL \keff on \uran mass.

\begin{figure}[h]
    \centering
    \includegraphics[width=3in]{../images/keff_vs_mass_235.png}
\caption{EOL \keff dependence on \uran mass}
\label{fig:eol_keff_vs_235_mass}
\end{figure}

This conclusion was useful for modeling a reactor's mass. Since EOL \keff is
dependent on \uran mass, for a fixed enrichment, EOL \keff can be modeled by
using the few parameters that affect reactor mass.

\subsection{Core Radius}
Core radius has a strong impact on EOL \keff. Fuel mass was directly correlated
to fuel fraction and core volume. For a fixed aspect ratio of 1, core volume was
a cubic function of core radius. This made EOL \keff strongly dependent on core
radius. Figure \ref{fig:eol_keff_vs_r_core} shows the correlation between 
core radius and EOL \keff. There was a strong relationship between core radius
and fuel mass, increasing fuel mass increased EOL \keff. Figure
\ref{fig:eol_keff_vs_r_core} also demonstrates the impact of fuel density on EOL
\keff. For a fixed mass of uranium, smaller, denser cores had more reactivity.

\begin{figure}[h]
    \centering
    \includegraphics[width=3in]{../images/keff_vs_mass_235_core_r.png}
\caption{EOL \keff dependence on core radius.}
\label{fig:eol_keff_vs_r_core}
\end{figure}

\subsection{Fuel Enrichment}
Fuel enrichment directly affects \uran mass in nuclear fuel. Higher enrichments
yield more \uran mass for a fixed mass of fuel. Figure
\ref{fig:eol_keff_vs_mass_enrich} shows how increasing enrichment increased \uran
mass and EOL \keff. For fixed \uran masses, increasing fuel mass also impacted
reactivity. For fixed \uran masses, higher enriched cores had greater \uran
density and as a result, higher EOL \keff.

\begin{figure}[h]
    \centering
    \includegraphics[width=3in]{../images/keff_vs_mass_235_enrich.png}
\caption{EOL \keff dependence on fuel enrichment}
\label{fig:eol_keff_vs_mass_enrich}
\end{figure}

\subsection{Fuel Pitch to Coolant Channel Diameter}
The fuel pitch to coolant channel diameter ratio (PD) also directly impacts
\uran mass in fuel. PD affects volumetric fuel fraction. Figure
\ref{fig:eol_keff_vs_PD_mass} shows the impact of PD on \uran mass and EOL
\keff. In addition to increasing \uran mass, PD impacts uranium density.
Reactors with larger PD ratios had greater \uran density which reduced neutron
leakage from the core and increased EOL \keff.

\begin{figure}[h]
    \centering
    \includegraphics[width=3in]{../images/keff_vs_mass_235_PD.png}
\caption{EOL \keff dependence on fuel pitch to coolant channel ratio}
\label{fig:eol_keff_vs_PD_mass}
\end{figure}

\subsection{Coolant Channel Radius}
EOL \keff is independent of coolant channel radius. Channel radius had no impact
on fuel mass and small geometric features had little importance for high 
energy neutrons. Figure \ref{fig:eol_keff_vs_r_cool} shows the impact of 
coolant channel radius on EOL \keff.

\begin{figure}[h]
    \centering
    \includegraphics[width=3in]{../images/keff_vs_cool_r.png}
\caption{EOL \keff dependence on coolant channel radius}
\label{fig:eol_keff_vs_r_cool}
\end{figure}

\subsection{Thermal Power}
EOL \keff is relatively independent of thermal power.
Figure \ref{fig:eol_keff_vs_power} shows the relationship between thermal power
and EOL \keff.

\begin{figure}[h]
    \centering
    \includegraphics[width=3in]{../images/keff_vs_power.png}
\caption{EOL \keff dependence on thermal power}
\label{fig:eol_keff_vs_power}
\end{figure}

As shown in Figure \ref{fig:eol_keff_vs_power}, EOL \keff was independent of core
thermal power. The depletion mass of uranium was negligigle compared to the mass 
of uranium required for the reactor to be critical at BOL. This was confirmed
with a quick calculation. Assuming a thermal power of 150 kW for 10 years and
192 MeV per fission, a reactor depletes approximately 600 grams of \uran.
This negligible compared to the BOL mass of a typical reactor
core (\~100-500 kg). Thermal power was not a strong predictor of EOL criticality.

\section{Summary}
The purpose of this work was to perform a rudimentry feature
selection, to aid the development of a reduced-order reactivity
model. The study helped identify key predictors for EOL \keff.

For this analysis, 3901 MCNP6.1 depletion calculations were
performed over a 5-dimensional sample space. Since thermal power had a
negligible impact on EOL \keff, a BOL excess reactivity target was deemed 
sufficient to ensure the reactor could operate for a 10 year lifetime. 
The other conclusion of this work was that, fissile
fuel mass had the greatest impact on reactivity. Core radius,
fuel fraction, and enrichment all impact fissile fuel mass and as a result, had
the greatest impact on reactivity.

The conclusions from this study were used to inform the development of the
surrogate reactivity model discussed in Chapter \ref{ch:crit_radius}. Fuel
enrichment was fixed at 93\% to maximize fissile fuel content. A constrained relationship
between core radius and fuel fraction would be used to define a core geometry with
sufficient excess reactivity. This relationship was used in the
thermal hydraulic model discussed in Chapter \ref{ch:mass_model}.
