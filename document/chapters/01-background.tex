\section{Previous Space Reactor Work}
Using nuclear technology to power space systems is not a new concept. Various
space programs have been using nuclear technology for decades to power
statellites and deep space probes.

\subsection{Radioisotope Thermoelectric Generators}
Traditionally, NASA has relied on Radioisotope Thermoelectric Generators to
produce electricity for deep-space and long term exploration missions.

\subsection{SP-100}

\subsection{Kilopower}
The Kilopower reactor is a near-term reactor prototype under development at Los
Alamos National Laboratory. The reactor was designed to achieve Technical
Readiness Level 5 by 2017 \citep{gibson_nasas_2017}. The Kilopower project is a
stirling engine system driven by a heat-pipe cooled core. The UMo core was
manufactured at the Y-12 National Security Complex from 93\%, High-Enriched
Uranium (HEU). Kilopower is a relevant reference because it fills a similar 
mission profile as the current project. NASA's target for Kilopower was a 1-10 kWe unit. 
The core was successfully tested from November 2017 to March 2018 \citep{poston_krusty_2018}. During the
tests, the UMo core was operated at 800 $\degree$C and 10 kWt power. While the
cooling and power cycle systems differ from a direct-brayton cycle reactor, the
Kilopower core design provided a good basis from which to design a direct-cooled
fast reactor.

\subsection{Other Fission Concepts}

\subsection { Nuclear Thermal Propulsion }

Nuclear thermal propulsion is a promising alternative to chemical rockets for
deep space missions. Nuclear thermal rockets improve the efficiency of
traditional chemical rockets to drastically decrease the travel time to planets
in the solar system. 
