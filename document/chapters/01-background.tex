\chapter{Background and Project Description}
Electrical power is vital for NASA to achieve its deep space exploration goals.
Electricity powers life support systems for astronaut habitats, communication
with Earth, and allows astronauts to perform experiments to help society better
understand the solar system. Long-life, high-output, compact electrical power
systems are vital to achieving future space exploration goals. It is crucial to
minimize the system mass when designing a terrestrail electrical system. Launch costs 
often restrict the design of systems launched
into space. Currently NASA is under contract with SpaceX to deliver supplies to
the International Space Station (ISS) in low-earth orbit (LEO). SpaceX delivers
supplies to the ISS for more than \$13,000 per kg\citep{spacex}. Launch costs to Mars or the
Moon will be significantly higher given the required energy to escape Earth's
orbit. To this end, a surrogate mass model of a small fission
nuclear reactor was developed. This work was a part of a larger
project to explore the mass tradeoff between the critical components of the
electrical power system in order to minimize the mass of the system as a whole.
This project developed the reactor mass model to support the goal of the overall
project.

\section{Scope}
This project supported a larger effort to explore the mass tradeoff between the
power cycle components and the reactor for a space power system. To this end,
this project focused on developing a surrogate model for reactor mass that could
be rapidly executed and easily integrated into an external power cycle model.
The reactor was modeled sufficiently to estimate its mass for a given set of
coolant flow conditions.
The purpose of this work was not to develop a fully-fledged reactor concept.
High-fidelity modeling of core neutronics and thermal hydraulics was not
performed for this work.

\section{Constraints}
The two main constraints for the power system were provided by the project sponsors,
and ultimately dictated by NASA. NASA has identified that: a 10 year lifetime 
and a 40 kWe power level, are sufficient to
support future terrestrial exploration throughout the solar system. All nuclear fuel
options were available for the reactor with a preference for high near-term
technical readiness. The use of high enriched uranium was acceptable. 
The reactor is direct-cooled, driving a
supercritical carbon dioxide brayton cycle to produce the desired electrical
output. The reactor will be used only for terrestrial applications, not in
transit. This helps shielding mass, because the reactor can be buried prior to
startup.

\section{Previous Space Reactor Work}
Using nuclear technology to power space systems is not a new concept. Various
space programs have been using nuclear technology for decades to power
statellites and deep space probes. Generally, nuclear power systems for space
exploration can be grouped into two categories, Radioisotope Thermoelectric
Generators (RTGs) and Fission Power Systems (FPS). 

\subsection{Radioisotope Thermoelectric Generators}
Traditionally, NASA has relied on RTGs to
produce electricity for deep-space and long term exploration missions. These
devices produce power for spacecraft traveling far from the sun, to parts of the
solar system where they could not rely on solar panels to meet their electrical
needs. Since 1961, NASA has deployed 46 RTGs on 27 different space missions
\citep{mmrtg_fact}.

One of the more common examples of a successful RTG design is the
General-Purpose Heat Source Radioisotope Thermoelectric Generator (GPHS-RTG).
GPHS-RTG relied on a \pu alpha source to generate heat. Modules of \pu
generators were stacked together for a combined thermal power of 4410
W$_{th}$. The
converter used silicon-germanium converter elements to produce 285 W of
electrical power at the beginning of mission (BOM).
\citep{bennett_mission_2006}. 

The GPHS-RTG design has been used on
numerous NASA deep space missions including; Galileo, Ulysses, New Horizons,
Cassini. All of the aforementioned spacecraft have had or continue to have
satisfactory power performance from their RTGs \citep{bennett_mission_2006}.
In addition to the GPHS-RTG, NASA has used similar RTG designs for other
missions. The Apollo moon missions, Viking and Curiosity Mars missions, Voyager
and Pioneer deep space probes have all used RTG technology to power their
instruments and communication systems. The Voyager probes continue to function,
sending back data to Earth as they push into interstellar space
\citep{mmrtg_fact}.

NASA has had great success using RTGs to power deep space missions. Many
RTG systems deployed decades ago are still powering deep space exploration
across the solar system. Despite their stellar performance record, RTGs are not
designed to provide large amounts of electrical power. The efficiency of RTGs is
too low for consideration at the 40 kWe power level. In addition the US supply of \pu
is dwindling, NASA can't afford to use \pu for a larger thermal system. As a
result, a fission power system (FPS), with more readily available fuel and higher conversion
efficiency (when coupled to a fluid cycle) was deemed more promising.

\subsection{Fission Power Systems}
Since RTGs have limited mass performance at high power levels, this project
focused on designing an
FPS to meet the mission requirements. Experience from past
fission designs was sought to help guide the design of the reactor. The choice
for fuel was based off of past designs. Some designs that were explored include;
SP-100, Kilopower, NTP and NEP concepts, and other solid-core design concepts.
These designs were used to constrain reactor fuel and geometry choices to
support the development of a surrogate mass model based on previous design
choices.
    
\subsubsection{SP-100}
    SP-100 was a concept fission power system design part of the
    Strategic Defense Initiative (SDI) of the United States. SDI was a defense program started by
    President Ronald Reagan's administration to defend the United States against intercontinental
    ballistic missiles from the Soviet Union. The cornerstone of the program was orbital systems to
    intercept the missiles before they re-entered the atmosphere. These systems
    had large electrical power requirements motivating the SP-100 project
    \citep{sp100}.

    SP-100 was a fission reactor coupled to thermoelectric converters that
    generated 100 kWe. The reactor was cooled with liquid lithium. Uranium
    nitride was chosen over \uox for its higher uranium density to allow a lower
    overall fuel mass. Pin type fuel assemblies were chosen with Nb-1Zr/Re
    cladding. Extensive material testing and irradiation was performed to
    verify the performance of the chosen fuels. While SP-100 was never built, it
    was a good example of a mature design concept developed for similar power
    applications. The project was not only tailored to defense needs, peaceful
    space exploration missions were also considered within the scope of the
    project.

    \subsubsection{Kilopower}
    The Kilopower reactor is a near-term reactor prototype under development at Los
    Alamos National Laboratory. The reactor was designed to achieve Technical
    Readiness Level 5 by 2017 \citep{gibson_nasas_2017}. The Kilopower project is a
    stirling engine system driven by a heat-pipe cooled core. The UMo core was
    manufactured at the Y-12 National Security Complex from 93\%, High-Enriched
    Uranium (HEU). Kilopower was a relevant reference because it filled a similar 
    mission profile as this project. NASA's target for Kilopower was a 1-10 kWe unit. 
    The core was successfully tested from November 2017 to March 2018 \citep{poston_krusty_2018}. During the
    tests, the UMo core was operated at 800 $\degree$C and 10 kWt power. While the
    cooling and power cycle systems differ from a direct-brayton cycle reactor, the
    Kilopower core design provided a good basis from which to design a direct-cooled
    fast reactor.

    \subsubsection { Nuclear Thermal Propulsion }

    Nuclear thermal propulsion (NTP) is a promising alternative to chemical rockets for
    deep space missions. Nuclear thermal rockets improve the efficiency of
    traditional chemical rockets to drastically decrease the travel time to planets
    in the solar system. NTP has been pursued by the United States Air Force
    (USAF) and NASA in the past to shorten journeys to Mars. 

    NTP works by heating a working fluid (hydrogen is most common) and
    accelerating it through a de Laval nozzle creating thrust. Chemical rockets
    heat the fluid through exothermic reactions. NTP engines rely on fission
    energy to heat the fluid. Using fission power as the thermal energy source means the engine does not require an
    oxidizing fuel to drive combustion.

    In order for NTP to be viable, it must achieve thrust-to-weight performance
    similar to chemical rockets. NTP achieves this with high temperatures, the
    higher the operating temperature, the more energy the working fluid absorbs
    and the greater specific impulse of the engine. Historically, the NERVA
    program focused on graphite-moderated cores \citep{webb_combined_2011}, but
    these reactors had poor material performance at high temperatures including
    cladding cracking. More recent work has used CERMET fuel to reach higher
    temperatures than previous designs. Uranium nitride suspended in a tungsten matrix
    can be pressed into blocks with coolant channels and can achieve remarkable
    power densities (9.34 GW/m3) allowing for feasible NTP performance
    \citep{webb_combined_2011}.

    \subsubsection{DoD Microgrid Concept}
    The United States Department of Defense (DoD) is pursuing a small fission
    reactor to support remote military installations by providing electrical
    power. Forward Operating Bases (FOBs) often struggle to find a stable
    electrical power source and the Department of Energy has directed national
    labs to leverage nuclear technology to solve this problem. The military
    identified 1 MWe as an electrical power target for the concept design
    \citep{army_reactor_slides}.
    Engineers at Los Alamos National Laboratory with experience in small core
    design worked to design a small fission core to meet these power needs. The
    reactor core was cooled with heat pipes and connected to a brayton cycle for
    power conversion.

    A monolithic core design was chosen for this project. There are multiple
    specific options for the design including separated fuel and coolant
    channels with fuel in pin form. Another option used dispersed fuel in a
    composite block with coolant channels. These cores offer durability because
    the fuel is stronger in block form.

    These monolithic designs were a good reference for a space reactor fuel.
    The monolithic designs are compact, durable and simple. Similar core designs
    can be found in numerous fission space concepts, including the recently
    tested Kilopower core.

    \subsubsection{INL Megawatt Concept Reactor}
    The Center for Space Nuclear Research at Idaho National Laboratory pursued a
    mass optimization for a megawatt-scale fission power system coupled to a
    high temperature brayton conversion cycle. The reactor was designed to
    support nuclear electric propulsion (NEP). NEP uses electricity from a
    reactor to ionize and accelerate a gas. Ion thrusters are extremely
    efficient and can be used for long-duration space flight. The advantage of
    NEP systems is the decoupling of the energy source and the propellant,
    similar to nuclear thermal propulsion. This decoupling reduces the amount of
    propellant required for a given mission.
    
    The proposed reactor design was fueled with a tungsten uranium-mononitride
    CERMET fuel. Like the NTP designs, CERMET fuel was explored because of its
    high thermal conductivity. The working fluid and coolant was helium. The
    reactor mass model was a combination of thermal hydraulic equations and
    critical mass requirements. MCNP was used to determine how much fuel mass
    was required for a given coolant channel radius. The result of the thermal
    and neutronic modeling was a analytic fit for the requried reactor mass as a
    function of reactor thermal power. The model was a simple linear fit to
    thermal power and did not account for mass effects driven by changing flow
    conditions (temperature, pressure, etc.) \citep{webb_combined_2011}. 
    
    The goal of the INL concept was similar to this project; to minimize the overall mass of the
    power system. In addition to a linear reactor mass model, the project
    developed simplified mass models for the power cycle components.
    The total optimization was performed by coupling the power cycle
    component mass models and exploring the tradeoffs between their performance
    and masses. The INL project was very similar to this work. This work will
    expand on the INL project by developing a model that responds to flow inputs
    (T, P, $\dot{m}$) where the INL reactor mass responded only to thermal
    power.

\section{Summary}
Since humans have pursued space exploration, there has been significant design, 
testing, and deployment effort of nuclear technology for spaceflight. Traditionally
RTGs have been used for deep space probes and terrestrial landers. Fission power
systems are being pursued for larger power needs such as terrestrial bases and
propulsion. Varying fuel and coolant concepts have been considered for fission
systems. Promising designs are using monolithic fuels, citing their fission
product retention and resistance to thermal expansion. Some projects, like
Kilopower have focused on deploying a system as soon as possible. To this end,
the Kilopower reactor design has been successfully tested at full fission power.
Other designs, such as the concept at INL have focused on mass optimization with
their power cycles. This project will expand on the efforts of the work at INL.
The power cycle and the reactor mass model coupling will be enhanced to include
flow conditions. In addition to coupling improvements, the entire optimization
will be performed with super-critical carbon dioxide. Fuel choices for the
reactor model will be informed from past design experience.

\section{Modeling Methodology}
This work developed a surrogate reduced-order model for the reactor component of
a small electrical system. The model rapidly estimates the minimum mass reactor 
for a given set of flow conditions, ensuring that it remains critical throughout
its lifetime and that fuel temperatures remain below safety thresholds.
Simplified analytical heat transfer and statistical neutronics functions 
were developed to model the minimum-mass reactor. Chapter
\ref{ch:inputs_constraints} discusses the flow inputs to the reactor model and
some initial design choices used to constrain the model. There were three main steps to
developing the reactor mass model: neutronics feature selection, reactivity
modeling, and thermal hydraulic modeling.

\subsection{Neutronics Feature Selection}
Chapter \ref{ch:sweeps} describes the feature selection process used to identify
which geometric, material, and operational parameters had the strongest impact
on reactivity. A large data set of depletion calculations was used to explore
the parameter space. The results of this work informed the developement of a
reduced-order reactivity model.

\subsection{Reactivity Modeling}
Chapter \ref{ch:crit_radius} discusses the development of a reduced-order
reactivity model that was used to constrain the reactor mass model to a target excess
reactivity. A combination of tri-linear interpolation and polynomial curve
fitting was used to model reactivity and constrain the core geometry to a target
reactivity.

\subsection{Thermal Hydraulic Modeling}
Chapter \ref{ch:mass_model} presents a root-finding algorithm to combine a
thermal hydraulic model with the reactivity constraint model. This constrained
thermal hydraulic model was used to identify a minimum-mass reactor that
satified flow conditions imposed by an external power cycle model.

\subsection{Reactor Design Verification}
The neutronic efficacy of the reactor mass model was verified with MCNP6.1. The
results of a test reactor mass optimization were turned into an MCNP model to
verify reactivity. Two reactivity verifications were performed; beginning of life
and end of life reactivity were checked to ensure they met the target
reactivity. In addition to the reactor design, the validity of the homogeneous
geometry approximations used to simplify MCNP modeling were verified. These
verifications are discussed in Chapter \ref{ch:verification}.

\subsection{Software and Data}
All neutronics modeling was performed using MCNP6.1 and ENDF7.2 nuclear data 
\cite{mcnp_citation}. Monte Carlo N-Particle (MCNP) code is a continuous-energy 
monte carlo simulation code developed at Los Alamos National Laboratory
for modeling radiation transport. MCNP uses random number generators and
probability density functions to model actual physics events like scattering,
absorption, and fission. MCNP tracks particles through a user-defined geometry
as they make their way from source locations to regions of interest. MCNP can
tally particles in space and energy, as well as their interaction rates with
surronding matter. In addition to tallying capabilities, MCNP is capable of
performing eigenvalue calculations. The \keff models in MCNP were used
extensively to predict reactivity response to input parameters. Finally, MCNP6.1
can model depletion in nuclear fuel. This capability was used to explore the
reactor response to burnup over its 10 year lifetime.

Statistical analysis was required to analyze the results of thousands of MCNP
simulations. Python3 was used to develop scripts to generate MCNP input files, parse data from output
files, filter and clean it for post processing, and to develop statistical curve
fits to neutronics results. These statistical models were combined with
analytical heat transfer models to model reactor response to power cycle flow
inputs.

