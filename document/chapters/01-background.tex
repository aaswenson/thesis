\section{Background}
Electrical power is vital for NASA to achieve its deep space exploration goals.
Electricity powers life support systems for astronaut habitats, communication
with Earth, and allows astronauts to perform experiments to help us better
understand the solar system. Long-life, high-output, compact electrical power
systems are vital to achieving future space exploration goals. Launch costs 
often restrict the design of systems launched
into space, currently NASA is under contract with SpaceX to deliver supplies to
the International Space Station (ISS) in low-earth orbit (LEO). SpaceX delivers
supplies to the ISS for more that \$55,000 per kg. Launch costs to Mars or the
Moon will be significantly higher given the required energy to escape Earth
orbit. As a consequence, when designing a terrestrial electrical system for
solar system exploration, it is crucial to minimize the mass of the system. This
project explored the mass tradeoff between the critical components of the
electrical power system in order to minimize the mass of the system as a whole.

\subsection{Scope}
The purpose of this work was not to develop a fully-fledged reactor concept.
High-fidelity core modeling of neutronics and thermal hydraulics was not
performed for this work.
This project focused on exploring the tradeoffs between reactor mass and the
mass of significant components of the power cycle. Traditional reactor physics
tools could not be used for depletion calculations because they are to slow to
integrate into an optmization routine. To support the optimizationo effort, this project focused on
developing a surrogate model for the reactor mass to be integrated into the
power cycle model. This model needed to take inputs from the power cycle and
rapidly exceute to return the minimum mass reactor that could meet the thermal
power requirement of the power cycle. The overall cycle was constrained to the
target lifetime and power outputs dictated by the project sponsors. The reactor
was modeled sufficiently such that future work could feasibly develop the
concept into a deployable reactor within a reasonable amount of time.

\subsection{Constraints}
The two main constraints for the system were provided by the project sponsors,
and ultimately dictated by NASA. NASA has identified the following performance
requirements: a 10 year lifetime and a 40 kWe power level, as sufficient to
support future terrestrial exploration throughout the solar system. All fuel
options were available for the reactor with a preferance for high near-term
technical readiness. High enriched uranium is acceptable. The reactor will be direct-cooled and coupled to a
supercritical carbon dioxide brayton cycle to produce the desired electrical
output. The reactor will be used only for terrestrial applications, not in
transit. This helps shielding mass, because the reactor can be buried prior to
startup.

\section{Previous Space Reactor Work}
Using nuclear technology to power space systems is not a new concept. Various
space programs have been using nuclear technology for decades to power
statellites and deep space probes. Generally, nuclear power systems for space
exploration can be grouped into two categories, Radioisotope Thermoelectric
Generators (RTGs) and Fission Power Systems (FPS). 

\subsection{Radioisotope Thermoelectric Generators}
Traditionally, NASA has relied on Radioisotope Thermoelectric Generators to
produce electricity for deep-space and long term exploration missions. These
device produced power for spacecraft traveling far from the sun, to parts of the
solar system where they could not rely on solar panels to meet their electrical
needs. Since 1961, NASA has deployed 46 RTGs on 27 different space missions
\citep{mmrtg_fact}.

One of the more common examples of a successful RTG design is the
General-Purpose Heat Source Radioisotope Thermoelectric Generator (GPHS-RTG).
GPHS-RTG relied on a \pu alpha source to generate heat. Modules of \pu
generators were stacked together for a combined thermal power of 4410 Wt. The
converter used silicon-germanium converter elements to produce 285 W of
electrical power at the beginning of mission (BOM).
\citep{bennett_mission_2006}. 

The GPHS-RTG design has been used on
numerous NASA deep space missions including; Galileo, Ulysses, New Horizons,
Cassini. All of the aforementioned spacecraft have had or continue to have
satisfactory power performance from their RTGs. \citep{bennett_mission_2006}.
In addition to the GPHS-RTG, NASA has used similar RTG designs for otehr
missions. The Apollo moon missions, Viking and Curiosity Mars missions, Voyager
and Pioneer deep space probes have all used RTG technology to power their
instruments and communication systems. The Voyager probes continue to function,
sending back data to Earth as they push into interstellar space
\citep{mmrtg_fact}  

NASA has experienced great success using RTGs to power deep space missions. Many
RTG systems deployed decades ago are still enabling deep space exploration
across the solar system. Despite their stellar performance record, RTGs are not
designed to provide large amounts of electrical power. The efficiency of RTGs is
too low for consideration at the 40 kWe level. In addition the US supply of \pu
is dwindling. NASA can't afford to use a \pu system at the 40 kWe level. As a
result, a fission power system was deemed appropriate for the project.

\subsection{Fission Power Systems}
Since RTGs are limited at high power levels, this project focused on designing a
fission power system to meet the mission requirements. Experience from past
fission designs was sought to help limit the design of the reactor. The choice
for fuel was based off of past designs. Some designs that were explored include;
SP-100, Kilopower, NTP concepts, and other solid-core design concepts.
    \subsubsection{SP-100}
    SP-100 was a concept fission power system design part of the United State's
    Strategic Defense Initiative (SDI). SDI was a defense program started by
    Ronald Reagan to defend the United States against the Soviet Union's intercontinental
    ballistic missiles. The cornerstone of the program was orbital systems to
    intercept the missiles before they re-entered the atmosphere. These systems
    had large electrical power requirements motivating the SP-100 project
    \ref{sp100}.

    SP-100 was a fission reactor coupled to thermoelectric converters that
    generated 100 kWe. The reactor was cooled with liquid lithium. Uranium
    nitride was chosen over \uox for its higher uranium density to allow a lower
    overall fuel mass. Pin type fuel assemblies were chosen with Nb-1Zr/Re
    cladding. Extensive material testing and irradiation was performed to
    verify the performance of the chosen fuels. While SP-100 was never built, it
    was a good example of a mature design concept developed for similar power
    applications. The project was not only tailored to defense needs, peaceful
    space exploration missions were also considered within the scope of the
    project.

    \subsubsection{Kilopower}
    The Kilopower reactor is a near-term reactor prototype under development at Los
    Alamos National Laboratory. The reactor was designed to achieve Technical
    Readiness Level 5 by 2017 \citep{gibson_nasas_2017}. The Kilopower project is a
    stirling engine system driven by a heat-pipe cooled core. The UMo core was
    manufactured at the Y-12 National Security Complex from 93\%, High-Enriched
    Uranium (HEU). Kilopower is a relevant reference because it fills a similar 
    mission profile as the current project. NASA's target for Kilopower was a 1-10 kWe unit. 
    The core was successfully tested from November 2017 to March 2018 \citep{poston_krusty_2018}. During the
    tests, the UMo core was operated at 800 $\degree$C and 10 kWt power. While the
    cooling and power cycle systems differ from a direct-brayton cycle reactor, the
    Kilopower core design provided a good basis from which to design a direct-cooled
    fast reactor.

    \subsubsection{Other Fission Concepts}

    \subsubsection { Nuclear Thermal Propulsion }

    Nuclear thermal propulsion (NTP) is a promising alternative to chemical rockets for
    deep space missions. Nuclear thermal rockets improve the efficiency of
    traditional chemical rockets to drastically decrease the travel time to planets
    in the solar system. NTP has been pursued by the United States Air Force
    (USAF) and NASA in the past to shorten journeys to Mars. 

    NTP works by heating a working fluid (hydrogen is most common) and
    accelerating it through a de Laval nozzle creating thrust. Chemical rockets
    heat the fluid through exothermic reactions. NTP engines rely on fission
    energy to heat the fluid. Using fission power as the thermal energy source means the engine does not require and
    oxidizing fuel to drive combustion.

    In order for NTP to be viable, it must achieve thrust-to-weight performance
    similar to chemical rockets. NTP achieves this with high temperatures, the
    higher the operating temperature, the more energy the working fluid absorbs
    and the greater specific impulse of the engine. Historically, the NERVA
    program focused on graphite-moderated cores \ref{webb_combined_2011}, but
    these reactors had poor material performance at high temperatures including
    cladding cracking. More recent work has used CERMET fuel to reach higher
    temperatures than previous designs. Uranium suspended in a tungsten matrix
    can be pressed into blocks with coolant channels and can achieve remarkable
    power densities (9.34 GW/m3) allowing for feasible NTP performance
    \ref{webb_combined_2011}.


