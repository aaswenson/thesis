\section{Conclusion}
The goal of this project was to develop a reactor mass model that did not rely
on traditional reactor physics tools. The model was designed to be integrated
into an overall power cycle mass model as part of a mass optimization process. A
framework was developed to model reactor mass as a submodule of an overall
power system mass model. This framework considered 

\section{Future Work}

\subsection{Design Exploration}
In order to constrain the reactor optimization space, two fuel types were
chosen. There are many other fuel types that could be considered. For monolithic
cores, exploration is simple as developing new critical radius curves and
applying different thermal and physical properties. Switching fuel form would be
more complicated. Different 1D heat transfer modeling would be required to
account for a different fuel geometry. The module nature of the reactor model
code would facilitate this work quite well. It would be interesting to explore
TRISO fuel as well as high temperature graphite fuel forms. These fuels are
capable of operating at extremely high temperatures, potentially increasing
efficiency of the power cycle.

\subsection{Modeling Work}
Reactor designs from this modeling workflow yield very rough models and limited
information about a design. The designs were meant to provide estimates of mass
to help an overall power cycle mass optimization. 
