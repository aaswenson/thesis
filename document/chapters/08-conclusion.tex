\chapter{Conclusion}
The goal of this project was to develop a reactor mass model that did not rely
on traditional reactor physics tools. The model was designed to be integrated
into an overall power cycle mass model as part of a mass optimization process. A
framework was developed to model reactor mass as a submodule of an overall
power system mass model. The modeling framework considered both thermal
hydraulic and neutronic requirements to optimize a minimum-mass reactor for a
fixed set of flow inputs. The reactor mass submodel was successfully integrated
into the power cycle model and was used to optimize the overall mass of the
power cycle. The resultant reactor was verified using traditional reactor
physics tools to be a valid design that, with significant time, funding, and
engineering effort, could be developed into a fully-fledged reactor concept.

\section{Future Work}
While the project was successful in setting up a workflow for modeling a nuclear
space reactor, there is much more work to be done. The workflow should be
applied to different designs to search for better mass performance and more
modeling should be done to improve the technical readiness of the designs.
Specifically, reactivity control and safety analysis needs to be incorporated
into the design process.

\subsection{Design Exploration}
In order to constrain the reactor optimization space, two fuel types were
chosen. While this limitation was useful for developing the model, 
there are many other fuel types that could be considered. For monolithic
cores, this exploration is simple as developing new critical radius curves and
applying different thermal and physical properties. Switching fuel form would be
more complicated. Different 1D heat transfer modeling would be required to
account for a different fuel geometry. It would be interesting to explore
TRISO fuel as well as high temperature graphite fuel forms. These fuels are
capable of operating at extremely high temperatures, potentially increasing
efficiency of the power cycle.

The reactor mass model and the supporting workflow was designed to support this
work. The geometry-specific calculations are modular and isolated in the thermal
hydraulic scripts. The core heat transfer functions could be modified to support
a new geometry. The neutronics modeling workflow was completely scripted in
Python, easily facilitating the replication of the workflow on a new geometry.

\subsection{Modeling Work}
Reactor designs from this modeling workflow yield very rough models and limited
information about a design. The designs were meant to provide estimates of mass
to help an overall power cycle mass optimization. Reactivity control and safety
analysis needs to be performed on any chosen design in order to advance the
design any further. Reactivity control will be essential to ensure the reactor
remains subcritical until reaching the mission destination. Once the power
system has been deployed, reactivity control systems will be essential for
reactor startup and any unexpected shutdowns during the system lifetime.
