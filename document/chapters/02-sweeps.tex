\section{Neutronic Constraints} \label{neutronic_sweeps}

A valid reactor design must meet certain neutronics constraints. The main
neutronics constraint is reactivity. The reactor must be able to sustain a chain
reaction from startup, to the final second of the 10 year lifetime. End of Life
(EOL) reactivity was the target metric for the neutronic design of the core. It
was important to determine which reactor parameters had the strongest impact on
EOL \keff. An n-dimensional sampling was performed to explore EOL \keff
dependence on various reactor parameters.

\subsection{Neutronics Parameters}
Homogeneous MCNP6.1 depletion models were used to analyze the EOL \keff response
to neutronics parameters (predictors). The core region was homogenized and
surrounded by a graphite reflector. A large set of 5-dimensional predictors and
EOL \keff results was produced to investigate the neutronics parameter space.
These predictors and their results helped develop an understanding of the
reactor response to important design and operational parameters. High Throughput
Computing capabilities at UW were used to perform 3901 depletion calculations
with MCNP6.1. Each model represents a unique sampling in every dimension using
the Latin Hypercube Sampling (LHS) technique. The LHS technique ensured even
sampling for every dimension. The sampled and fixed dimensions/parameters are
shown in Table \ref{tab:lhs_sweep_vars}.

\begin{table}[h]
  \centering
  \caption{Homogeneous Geometry and Depletion Parameters}
  \begin{tabular}{ll}
    \toprule
     Core Radius                		   & 10 - 50 [cm] \\
     Fuel Enrichment 					   & 20\% - 90\% $^{235}$U\\
     Reflector Thickness				   & 15 [cm]\\
     Coolant Channel Radius                & 0.5 - 1 [cm] \\
     Fuel Pitch to Coolant Channel D.      & 1.1-1.6 [-]\\
     Thermal Power						   & 80-200 [kW]\\
     Core Aspect Ratio					   & 1 [-] \\
     Fuel Temp  						   & 300 [K]\\
     Reactor Physics Code, Data			   & MCNP6.1, ENDF-7.2
  \end{tabular}
  \label{tab:lhs_sweep_vars}
\end{table}

\subsection{Neutronics Sampling Results}
The target metric for the neutronics sampling was an EOL \keff equal to one. In
order to determine the dependence of EOL \keff on each swept parameter in Table
\ref{tab:lhs_sweep_vars}, EOL \keff was plotted against the parameters.

\subsubsection{Core Radius}
Fuel mass is the strongest predictor of EOL \keff. As a reseult, core radius has a strong impact on EOL \keff. Fuel mass is directly correlated
to fuel fraction and core volume. For a fixed aspect ratio of 1, core volume is
a cubic function of core radius. This makes EOL \keff strongly dependent on core
radius. The volume to surface area ratio is a positive function of core radius
hence, neutron leakge is reduced in larger cores. 

\begin{figure}[h]
    \centering
    \includegraphics[width=3in]{../images/keff_vs_core_r.png}
\caption{EOL \keff dependence on core radius.}
\label{fig:eol_keff_vs_r_core}
\end{figure}`

\subsubsection{Fuel Enrichment}
EOL \keff is directly dependent on fuel enrichment. Mass of \uran is the most
important metric to predict EOL \keff. Figure
\ref{fig:eol_keff_vs_enrich} shows the result of this sweep in the context of
fuel mass dependence, reducing core radius and fuel pitch to channel diameter to
one metric, total fuel mass. The colorbar shows the impact of enrichment. For a fixed core
radius, increasing enrichment leads to increasing EOL \keff. 

\begin{figure}[h]
    \centering
    \includegraphics[width=3in]{../images/keff_vs_mass_enrich.png}
\caption{EOL \keff dependence on fuel enrichment}
\label{fig:eol_keff_vs_mass_enrich}
\end{figure}`


\subsubsection{Coolant Channel Radius}
EOL \keff is independent of coolant channel radius. Small geometric features
have little importance for high energy reactors. Figure
\ref{fig:eol_keff_vs_r_cool} shows the result of this sweep.

\begin{figure}[h]
    \centering
    \includegraphics[width=3in]{../images/keff_vs_cool_r.png}
\caption{EOL \keff dependence on coolant channel}
\label{fig:eol_keff_vs_r_cool}
\end{figure}`

\subsubsection{Fuel Pitch to Coolant Channel Diameter}
Fuel pitch to coolant channel diametre was used inverse of the traditional pitch
to diameter ratio in power reactors. The inverted fuel configuration with
coolant flowing through the middle of a fuel block motivates this metric. Like
core radius, PD impacts EOL \keff by dictating fuel mass. The high-energy
neutron spectrum means the geometric features have a small impact on the
neutronics of the reactor. A large PD yields a large fuel fraction and as a
result, a large fuel mass. Figure
\ref{fig:eol_keff_vs_PD_mass} shows the result of this sweep, with the colorbar
showing total fuel mass.

\begin{figure}[h]
    \centering
    \includegraphics[width=3in]{../images/keff_vs_PD_mass.png}
\caption{EOL \keff dependence on fuel pitch to coolant channel ratio}
\label{fig:eol_keff_vs_PD_mass}
\end{figure}`


\subsubsection{Thermal Power}
Initially, the result of greatest interest was the mass dependence on thermal
power because thermal power was the coupling parameter with the power cycle.
Figure \ref{fig:eol_keff_vs_power} shows the relationship between thermal power
and EOL \keff.

\begin{figure}[h]
    \centering
    \includegraphics[width=3in]{../images/keff_vs_power.png}
\caption{EOL \keff Power Dependence}
\label{fig:eol_keff_vs_power}
\end{figure}

As shown in Figure \ref{fig:eol_keff_vs_power}, EOL \keff is independent of core
thermal power. This result is not surprising considering the depletion rates in
the core. Assuming 1 MWd/gU is an achievable burnup, the reactor depletes
approximately 1000 kg of uranium at 200 kW of thermal power. The depletion mass
of uranium is negligigle compared to the mass of uranium required for the
reactor to be critical at BOL. Thermal power is not a strong predictor of EOL
criticality.

\subsubsection{Uranium 235 Mass}
The strongest predictor of EOL \keff is a combination of some of the above
swept parameters. The most important metric for EOL \keff is the mass of \uran
in the system. Figure \ref{fig:eol_keff_vs_235_mass} shows the dependence of EOL
\keff on \uran mass.

\begin{figure}[h]
    \centering
    \includegraphics[width=3in]{../images/keff_vs_mass_235.png}
\caption{EOL \keff \uran mass dependence}
\label{fig:eol_keff_vs_235_mass}
\end{figure}

This conclusion is useful for modeling a reactor's mass. Since EOL \keff is
dependent on \uran mass, for a fixed enrichment, EOL \keff can be modeled 
solely with reactor mass.


